\documentclass[10pt,a4paper]{report}
\usepackage{xetexko}
\usepackage{siunitx}
\sisetup{inter-unit-product =$\cdot$}

\title{물리 노트\\(생각하며 배우는 대학물리학 - 이기영 지음)}
\author{tsshin1985@gmail.com}
\date{February 2018}

\begin{document}
	
	\maketitle
	
	\part{힘과 운동}
	
	\chapter{우리 물질 세계}
	
	\section{매우 큰 스케일의 세계}
	
	\section{매우 작은 스케일의 세계}
	
	\section{물리학이란 무엇인가?}
	
	\chapter{운동의 기술 (Description of Motion)}
	
	\section{과학의 시작 : 관성의 법칙}
	
	아리스토텔레스 : ``일정한 운동 상태를 유지하려면 일정한 힘을 계속 줘야 한다.''\\
	갈릴레오 : 사고실험을 통해, ``일단 운동하던 물체는 마찰력만 없다면 영원히 운동을 계속 한다.''
	(마찰\footnote{갈릴레이의 사고실험 더 찾아보기 (마찰 혹은 마찰력을 알고 있었나?)}이 없다면 운동 상태는 유지된다.)\\
	$\rightarrow$ 관성의 법칙 (Law of Inertia) = 뉴턴의 운동 제1법칙\\
	$\rightarrow$ 운동상태를 변하게 하는 요인이 바로 `힘'이다.\footnote{혹은, 운동상태를 변하게 하는 요인을 `힘'이라고 하자?}\\
	$\rightarrow$ 힘이란 물체의 운동상태를 변화시키는 요인이다. (힘의 정의 도출)\\
	\\
	어떤 물체가 운동할 때, 관측자에 따라 그 운동이 다르게 보인다.\\
	$\rightarrow$ 관측자가 운동을 기술하는 좌표계를 `기준계'라고 하며, 관성의 법칙이 성립하는 기준계를 `관성기준계'라고 한다.
	한 관성기준계에서 볼 때 일정한 속도로 운동하는 관측자의 기준계도 관성기준계이다.
	
	\section{물리량과 그 측정}
	
	\section{위치벡터와 변위벡터}
	
	\section{빠르기와 속도}
	
	\section{직선운동에서의 속도}
	
	\section{가속도}
	
	\section{등가속도 운동}
	
	\chapter{특수상대성 이론}
	
	\section{우주의 중심과 상대속도}
	
	\section{특수상대성 이론}
	
	\section{로렌츠 변환}
	
	\chapter{힘과 운동}
	
	\section{뉴턴의 운동 제2법칙}
	
	힘이 작용하면 물제의 운동상태가 변한다. (뉴턴의 운동 제1법칙) $\rightarrow$ `어떻게' 변하는가?\\
	뉴턴 :\\
	(1) 일정한 힘이 작용하는 경우, 가속도는 일정하다. (가속도 개념의 도입)\\
	(2) 이 때 가속도와 힘의 크기는 비례한다.\\
	$\rightarrow$ 뉴턴의 운동 제2법칙 또는 줄여서 뉴턴의 운동 법칙\\
	서로 일정한 속도로 운동하는 두 기준계에서 측정한 물체의 속도는 서로 다르지만, 가속도는 똑같다 $\rightarrow$ 힘이 똑같다 (by 3장)\\
	뉴턴은 a$\propto$F에 비례상수를 도입\\
	$a=\frac{1}{m}F$\\
	$\rightarrow$ m : 자신의 운동상태가 변하는 것에 저항하는 물체 고유의 특성\\
	$\rightarrow$ 질량(mass)이라고 이름 붙임 (by 뉴턴), `관성의 척도'라고도 한다.\\
	힘의 단위 $\SI{1}{\newton} = \SI{1}{kg.m/s^{2}}$\\ % 에러 발생
	
	\section{무게와 질량}
	
	뉴턴의 만유인력 발견, 중력 (지구), 중력가속도 (g), 무게 (mg, 중력의 세기)\\
	관성질량 : $\frac{a_2}{a_1} = \frac{m_1}{m_2} = k \Rightarrow m_1 = k m_2$ (관성의 상대적 크기를 이용한 측정)\\
	중력질량 : $\frac{a_2}{a_1} = \frac{g}{g} = 1, m_1 = m_2$ (무게비교를 통한 측정)\\
	관성질량과 중력질량이 같다.
	
	\section{등가성원리}
	
	아인슈타인 : ``중력이 왜 질량에 비례해야 하지?" ``중력은 가속되는 계의 관성력과 완전히 동등하다.''
	
	\section{여러가지 힘}
	
	수직항력(전기적반발력), 부력(전기적반발력), 마찰력, 탄성력(=복원력, 후크의 법칙, 용수철 상수), 전기력, 자기력
	
	\section{힘의 합력과 기본 힘}
	
	힘 $F_1, F_2$가 동시에 작용한다면, $F = F_1 + F_2$가 작용할때의 운동과 같다. $F$를 합력이라고 한다. $\sum F = 0$을 만족한다면 평형상태라고 한다.\\
	\\
	기본힘 : 중력, 전기력, 강력(핵력), 약력
	
	\section{뉴턴의 운동 제2법칙의 적용}
	
	운동을 기술하는 단계\\
	(1) 질량이 있는 모든 물체들을 파악\\
	(2) 각 물체에 작용하는 힘들을 파악\\
	(3) 각 물체에 운동방정식 적용\\
	\\
	수직으로 던진 공, 연결된 두 물체, 도르래에 연결된 물체, 경사면에서 미끄러지는 물체
	
	\section{마찰력}
	
	정지마찰력, 최대정지마찰력, 운동마찰력, 정지마찰계수, 운동마찰계수, 정지거리
	
	\section{저항력과 마찰력}
	
	저항력, 종단속도(=끝속도)
	
	\section{뉴턴의 운동 제3법칙}
	
	작용-반작용의 법칙 : 어떤 종류의 힘이든 \textbf{두 물체간의} 힘은 `상호작용'형태로 작용한다.\\
	한 물체에 작용하는 중력과 수직항력은 작용-반작용 관계가 아니다.\\
	지구가 물체를 당기는 중력과 물체가 지구를 당기는 힘이 작용-반작용이다.\\
	``말이 마차를 끌 때 마차도 같은 크기의 힘으로 말을 당기므로 말은 마차를 끌 수 없다.'' $\rightarrow$ 어떻게 답해야 하지?
	
	\chapter{공간에서의 운동}
	
	\section{지표면 근처의 공간운동}
	
	공 던지기
	
	\section{등속원운동에서의 가속도}
	
	\section{만유인력의 법칙}
	
	만유인력과 전기력의 특성 : 입자를 점으로 봐도 동일하다.
	
	\section{인공위성의 운동}
	
	인공위성의 운동, 바닷물의 조수 현상
	
	\section{관성력}
	
	관성력 = 가짜힘\\
	가속되지 않는 계, 즉 관성력이 존재하지 않는 기준계를 관성기준계라고 한다.\\
	원심력, 코리올리의 힘
	
	\part{에너지와 운동량 (Energy and Momemtum)}
	
	\chapter{에너지와 일}
	
	\section{에너지}
	
	에너지란, `운동할 수 있게 하는 그 무엇'\footnote{가능한 일의 양?}
	
	\section{일(work)}
	
	
\end{document}
