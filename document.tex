\documentclass[10pt,a4paper]{report}
\usepackage{xetexko}

\title{물리 노트 (생각하며 배우는 대학물리학 - 이기영 지음)}
\author{tsshin1985@gmail.com}
\date{February 2018}

\begin{document}
	
	\maketitle
	
	\part{힘과 운동}
	
	\chapter{우리 물질 세계}
	
	\section{매우 큰 스케일의 세계}
	
	\section{매우 작은 스케일의 세계}
	
	\section{물리학이란 무엇인가?}
	
	\chapter{운동의 기술 (Description of Motion)}
	
	\section{과학의 시작 : 관성의 법칙}
	
	아리스토텔레스 : ``일정한 운동 상태를 유지하려면 일정한 힘을 계속 줘야 한다.''\\
	갈릴레오 : 사고실험을 통해, ``일단 운동하던 물체는 마찰력만 없다면 영원히 운동을 계속 한다.''
	(마찰\footnote{갈릴레이의 사고실험 더 찾아보기 (마찰 혹은 마찰력을 알고 있었나?)}이 없다면 운동 상태는 유지된다.)\\
	$\rightarrow$ 관성의 법칙 (Law of Inertia) = 뉴턴의 운동 제1법칙\\
	$\rightarrow$ 운동상태를 변하게 하는 요인이 바로 `힘'이다.\footnote{혹은, 운동상태를 변하게 하는 요인을 `힘'이라고 하자?}\\
	$\rightarrow$ 힘이란 물체의 운동상태를 변화시키는 요인이다. (힘의 정의 도출)\par
	
	어떤 물체가 운동할 때, 관측자에 따라 그 운동이 다르게 보인다.\\
	$\rightarrow$ 관측자가 운동을 기술하는 좌표계를 `기준계'라고 하며, 관성의 법칙이 성립하는 기준계를 `관성기준계'라고 한다.
	한 관성기준계에서 볼 때 일정한 속도로 운동하는 관측자의 기준계도 관성기준계이다.
	
	\section{물리량과 그 측정}
	
	\section{위치벡터와 변위벡터}
	
	\section{빠르기와 속도}
	
	\section{직선운동에서의 속도}
	
	\section{가속도}
	
	\section{등가속도 운동}
	
	\chapter{특수상대성 이론}
	
	\section{우주의 중심과 상대속도}
	
	\section{특수상대성 이론}
	
	\section{로렌츠 변환}
	
	\chapter{힘과 운동}
	
	\section{뉴턴의 운동 제2법칙}
	
	힘이 작용하면 물제의 운동상태가 변한다. (뉴턴의 운동 제1법칙) $\rightarrow$ `어떻게' 변하는가?\\
	뉴턴 :\\
	(1) 일정한 힘이 작용하는 경우, 가속도는 일정하다. (가속도 개념의 도입)\\
	(2) 이 때 가속도와 힘의 크기는 비례한다.\\
	$\rightarrow$ 뉴턴의 운동 제2법칙 또는 줄여서 뉴턴의 운동 법칙
	
\end{document}
